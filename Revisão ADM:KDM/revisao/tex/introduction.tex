A possible definition for ``software concern'' is anything which stakeholders regard as a conceptual unit~\cite{Eaddy}. Examples of common software concerns include Persistence, Caching, Synchronization among others~\cite{valterWASP1}. Developers and architects are continuously in need of up-to-date knowledge about the concerns currently implemented in their legacy system, and about the location of these concerns throughout the code. For example, during maintenance and reengineering, when there are bugs to be fixed, the maintenance task affects the whole implementation of a concern, and possibly to other concerns with which the fixed concern interacts.

Mining techniques for crosscutting concerns are indispensable for software maintenance, reverse engineering, reengineering and even for re-documentation~\cite{Marin2007}. However, manually applying a mining technique for crosscutting concern is difficult and error-prone. This came about because, legacy systems tend to: (\textit{i}) have complex architectures with several clones spread out throughout the source code, (\textit{ii}) involve several kinds of crosscutting concerns, e.g., patterns, architectural styles, business rules and non-functional properties and (\textit{iii}) be very large, making the manual mining impractical. Thus, there is a need to use techniques and fully or semi-automated tools, which can aid software engineers to locate crosscutting concern into the legacy systems. In this context, the research area which aims to investigate techniques and tools to improve the mining of crosscutting concerns is known as ``concern mining".

The aim of this paper is threefold. First, we aim to identify techniques employed in the research area herein described. Therefore, we have carried out a systematic review identifying  mining techniques for crosscutting concerns. Second, we intent to extend the taxonomy presented by Kellens et al.~\cite{Kellens}. Thus, we selected 62 papers and among them, we identified 18 mining techniques for crosscutting concerns. This taxonomy was proposed in 2007 and it needs to be updated because we found 7 new techniques developed in the past few years. Third, we recommend possible combinations of these techniques/tools that might improve recall and precision metrics for both Persistence and Observer concerns. We recommend four combinations of techniques discovered herein in order to improve recall and precision for Persistence concern. Similarly, we proposed two initial combinations of techniques identified to make better recall and precision metrics for Observer concern. This combination can be a motivation for potential users to improve the identification of well-known concerns.


%The aim of this paper is to  identify a large amount of techniques related to crosscutting concern mining. Therefore, we have carried out a systematic review identifying  mining techniques for crosscutting concern in the literature. As a consequence, we have discovered that in the last years a considerable number of mining techniques for crosscutting concern are available. Another aim of this paper is to present a roadmap of possibles combinations of the discovered techniques based on the precision and recall metrics. This combination can be a motivation for potential users  to identify remaining open research questions, possible avenues for future research and to decide  which techniques are most suitable to improve the identification of well-known concerns. Another motivation is that previous studies - described in Section~\ref{related} - the authors realized a review without follow a systematic process in the foregoing research area, as consequence, some techniques could be left out in their review. Moreover, the authors devised a comparative framework and taxonomy which allowed one to discriminate among the different techniques, however, a set of new techniques has been developed by combining existing approaches or creating it from the scratch. Thus, we argue that the framework proposed by~\cite{Kellens} must be extended in order to fulfill these new techniques.

% This systematic review shows that there are a lot of interesting and important research topics that could be investigated yet. In addition given the techniques identified we also recommend possible combination of them based on the precision and recall metrics to identify remaining open research questions and possible avenues for future research.

%The main motivation is to identify new opportunities for research on mining techniques for crosscutting concern and to devise a roadmap to potential users of mining techniques for crosscutting concerns, to assist potential users in deciding  which techniques are most suitable to improve the identification of well-known concerns.


%. Therefore, we have carried out a systematic review identifying all the available mining techniques for crosscutting concern in the literature. Thus, by using this review others researches can identify which mining techniques for crosscutting concern exists, which ones is more used and gather information regard which technique accomplish the needs of their projects. Another motivation is that previous studies - described in Section~\ref{related} - the authors realized a review without follow a systematic process in the foregoing research area. Consequently, some techniques could be left out in their review.

This paper is organised as follows: In Section~\ref{method} presents how we have planned, conducted, reported and validated the systematic review. In addition, in this section there is also an extend taxonomy which was firstly proposed by Kellens et al.~\cite{Kellens}. In Section~\ref{threats} there are the threats to validity of our study. Section~\ref{related} presents a related work. Concluding remarks are made in Section~\ref{conclusion}.