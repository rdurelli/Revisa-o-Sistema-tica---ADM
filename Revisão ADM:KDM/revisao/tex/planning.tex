In this phase we have defined the review protocol. This protocol contains: (\textit{i}) the research questions, (\textit{ii}) the search strategy, (\textit{iii}) the inclusion and exclusion criteria and (\textit{iv}) the data extraction and synthesis method.

Research questions must embody the review study purpose. Moreover, these questions reflect the general scope of the review study. The scope is comprised of population (i.e., population group observed by the intervention), intervention (i.e., what is going to be observed in the context of the planned review study), and outcomes of relevance (i.e., the results of the intervention). Furthermore, during the conduction of this step, it was also necessary to establish the scope of the review study. According to the systematic review process~\cite{Kitchenham}, the scope has to be established using the PICO criteria. Thus, herein our \textbf{Population} is published scientific literature reporting on the use of Architecture-Driven Modernization and its metamodels. The \textbf{Intervention} is published scientific literature interested with Architecture-Driven Modernization and its metamodels. The \textbf{Comparison} is not applied herein. Finally, the \textbf{Outcomes of relevance} is an overview of the studies that have been conducted in the field of Architecture-Driven Modernization and its metamodels, emphasizing primary studies that report on the process used in the research area, from observing such an aggregated data set, we also intend to provide insight into the frequencies of publication over time to inspect trends.   

%\begin{itemize}

%\item \textbf{Population:} published scientific literature reporting on some existing mining techniques for crosscutting concern.

%\item \textbf{Intervention:} published scientific literature concerned with mining techniques for crosscutting concern. Furthermore, we also aim at determining which techniques are the most used within academic settings.

%\item \textbf{Comparison:} No applied herein.

%\item \textbf{Outcomes of relevance:} an overview of the studies that have been conducted in the field of crosscutting concern mining, emphasizing primary studies that report on the techniques used in the research area. From observing such an aggregated data set we also intend to provide insight into the frequencies of publication over time to inspect trends.

%\end{itemize} 

%Como ADM esta sendo aplicada na literatura para auxiliar a modernizacao de sistemas legados

As described before, the objective of this review is to find out \textbf{How ADM has been applied in the literature to assist engineers during the process of modernization of legacy systems?} In order to achieve such objectives we worked out four research questions. The questions are:

\begin{description}

\item[\textbf{RQ$_1$:}] What are the focus area most discussed and least discussed in the literature regarding the ADM? Moreover, what types of contributions have been presented so far?

\item[\textbf{RQ$_2$:}]  Given the ADM's standards metamodels, which one has been more used in the literature?


\item[\textbf{RQ$_3$:}]  Which research types have been employed into the field herein?

\item[\textbf{RQ$_4$:}]  Which avenues are often used to publish research related to ADM?

%\item[\textbf{RQ$_4$:}] What different types of research are presented in literature and what extent has been addressed? 



%\item[\textbf{RQ$_4$:}] Given a set of concerns, which are the most indicated techniques for performing the mining?

%\item[\textbf{RQ$_5$:}] How can someone combine the techniques for improving the precision and recall metrics?

%\item[\textbf{RQ$_3$:}]  Considering the techniques that we found, which ones have automated support?

\end{description}

To address \textbf{RQ$_1$}, we read all primary studies in order to identify the topic of each study. Next, we arrange all studies according to them topics. Whether any kind of disagreement related to the topic that the article meets, the article was marked and was discussed with everyone involved in the review in order to clarify which topic it belongs. With respect to \textbf{RQ$_2$}, we also read all primary studies and identified which metamodel was used in the study. Concerning to \textbf{RQ$_3$}, we gathered all references of each primary studies herein. Afterwards, we organized them among journal, conference, and workshop. Finally, with respect to \textbf{RQ$_4$}, we used and adapted the scheme proposed by Wieringa et al~\cite{Wieringa:2005:REP:1107677.1107683} in order to classify each primary studies into a research type, see Section~\ref{research_type}.


Afterwards, we have defined the search string and chosen the electronic databases. The search string was created based upon the following keywords: \textit{Architecture Driven Modernization, Architecture-Driven Modernization, ADM, Object Management Group, OMG, Legacy Systems, Knowledge Discovery Metamodel, KDM, Abstract Syntax Tree Metamodel, ASTM, Structured Metrics Metamodel - SMM, Model-Driven Development}. A sophisticated search string was constructed using boolean operators i.e., \textit{AND}, \textit{OR} and \textit{NOT}. Figure~\ref{search_string} shows the search string elaborated. The search have encompassed electronic databases which are deemed as the most relevant scientific sources~\cite{Dyba} and therefore likely to contain important primary studies. We have used the search string on the following electronic databases: \textit{ACM}, \textit{IEEEXPLORE}, \textit{Scopus}, \textit{Web of Science} and \textit{Engeneering Village}. Note that since the features provided by various databases as well as the exact syntax of search strings to be applied vary from one database to other, the string given in Figure~\ref{search_string} was actually used to construct a semantically equivalent string specific to each database.

%Furthermore, no limits were placed on date of publication with a view to not restrict the review study scope. %Aimed at keeping track of the selected papers, we used JabRef\footnote{http://jabref.sourceforge.net/}, an open source system for bibliography reference management. 

\begin{figure}[!h]
\centering
  % Requires \usepackage{graphicx}
  \includegraphics[scale=0.35]{figuras/SearchStringADM}
\caption{Search String.}
\label{search_string}
\end{figure} 

Then, in order to determine which primary studies are relevant to answer our research questions, we have applied a set of inclusion and exclusion criteria. Inclusion criteria devised and applied are:

\begin{enumerate}[(a)]%for small alpha-characters within brackets.
\item \textbf{The primary study presents at least one solution of modernization by means of ADM:} the paper provides evidences that the ADM assists the software engineer during the modernization/refactoring of legacy system.

\item \textbf{Studies that explicitly present an ADM approach:} the paper provides an approach to assist the software engineer to modernize the legacy system.

\item \textbf{The primary study presents at least one type of evaluation technique for ADM:} without the results of the evaluation we would not be able to make comparisons desired. In other words, the paper must clearly present which assessment techniques have been employed to evaluate the ADM process, i.e., case study, experiment, survey, etc.
%\textbf{IC$_1$:} the primary study presents at least one mining technique for crosscutting concern
\end{enumerate}

Not all of these criteria must be present for every primary study. However, at least the former, i.e., (a), must be present. If all criteria were mandatory, the number of selected techniques would decrease significantly.

Exclusion criteria devised and applied are:
\begin{enumerate}[(a)]

\item \textbf{Papers which mentioned ADM and its metamodels in the abstract only:} this was required because we found many studies that mentioned ADM and its metamodels in their opening sentences as a principal concept, however, the studies did not really address it.

\item \textbf{Papers that present only recommendations, guidelines or principles:} usually this kind of papers do not present practical approach to modernize legacy systems.

\item \textbf{Introductory papers for books and workshops.}

\item \textbf{The primary study is a short paper:} papers with two pages or less were not considered herein, since we considered that this kind of study do not own sufficient information.

\item \textbf{Papers from industrial conferences, and non English publications.}
 
\end{enumerate}

% \textbf{IC$_1$:} the primary study presents at least one mining technique for crosscutting concern and \textbf{IC$_2$:} the primary study presents at least one type of evaluation technique for crosscutting concern. And the following exclusion criteria: \textbf{EC$_1$:} the primary study presents data mining technique, however, such technique is applied to databases and not for crosscutting concern mining and \textbf{EC$_2$:} the primary study is a short paper (papers with twos pages or less).

%\begin{description}

%\item[\textbf{IC$_1$:}] The primary study presents at least one mining technique for crosscutting concern

%\item[\textbf{IC$_2$:}] The primary study presents at least one type of evaluation technique for crosscutting concern.

%\end{description}

%and the following exclusion criteria:

%\begin{description}

%\item[\textbf{EC$_1$:}] The primary study is not about mining techniques for crosscutting concern.

%\item[\textbf{EC$_2$:}] The primary study presents data mining technique. However, such technique is applied to databases and not for crosscutting concern mining.

%\item[\textbf{EC$_3$:}] The primary study is not available in an electronic format.

%\item[\textbf{EC$_3$:}] The primary study is a short paper (papers with twos pages or less).

%\item[\textbf{EC$_5$:}] The primary study is written neither english nor portuguese.

%\end{description}

We devised data extraction forms to accurately record the information obtained by the researchers from the primary studies. The form for data extraction provides some standard information, such as (\textit{i}) a brief of the primary study, highlighting where ADM and its metamodels are used, (\textit{ii}) date of data extraction, (\textit{iii}) title, authors, journal, publication details and (\textit{iv}) a list of each conclusion and statement encountered for each question. 

During the extraction process, the data of each primary study were independently gathered by three reviewers. The review was performed in August, 2013 by two M.Sc. and a Ph.D. student; the achieved results were crossed and then validated. All the results of the search process are documented in the web material\footnote{http://tinyurl.com/99spmaz}. Therefore, it is clear to others how thorough the search was, and how they can find the same documents.
