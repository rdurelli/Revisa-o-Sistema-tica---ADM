Recent proposals in ADM have focused mainly on providing approaches to modernize legacy system to another plataform/architecture resulting in appearance of several proposals, see Section~\ref{ssub:approach}. However, if we look at overall problem of the integration of modernization into an ADM context, there is still a significant work to be done. Following the ADM approach, models are the main focus for visualizing an executable view of the system and are also used to modernize a system. Therefore, in order to integrate ADM's metamodels into this larger context, the area of discovering knowledge, i.e., parsers, needs more attention along with solution to verification of models. Very few works have been reported on in the literature (e.g.,~\cite{5440163,Bruneliere:2010:MGE:1858996.1859032}) so far that have addressed parsers to represent instances of KDM, but even these parsers provide limited infrastructure to represent all KDM's layers. Thus, we argue that new researches must be conducted to create a complete parser in order to represent all KDM's layer. Besides, the discovering of knowledge are often mostly static in a sense that they are unable to obtain knowledge during the executing of the target legacy system. In~\cite{5871783} the authors presented an approach that dynamically discovery knowledge of the targe system during its execution. Nevertheless, this approach lacks support for complex discovery of knowledge once it is just based on the KDM's event metamodel package. Hence further research is required to boost the discovering of knowledge, e.g., use the knowledge dynamically identified combined with other models of KDM as the code and data models in order to acquire more meaningful models, since these models make it possible to consider additional sources of embedded knowledge.

In addition, we observed that there are three main hurdles in need to be more researched so that modernization techniques can be used in the KDM specification in an effective and widespread way. The first hurdle is the present lack of a fully developed idea of ``good'' KDM style. This is an important issue, for a clear notion of style is a fundamental prerequisite for the use of modernization, enabling software engineers to see where they are heading when modernizing their KDM model. Fowler et al.~\cite{refactImpro} advocated a specific notion of style for object-oriented programming through a catalog of 22 code smells, compounded by a catalog of 72 refactorings through which those smells can be removed from existing code. These catalogs proved very useful in bringing the concepts of refactoring and good object-oriented style to a wider audience and in providing programmers with guidelines on when to refactor and how best to refactor. A second one - both a cause and a consequence of the first - is the present lack of a KDM equivalent of such catalogues. We assume that the process of modernization by using KDM would equally benefit from KDM specific catalogues of smells and refactorings, helping software engineers to detect situations where the KDM model could be improved and guiding them through the corresponding transformation processes. A third hurdle is the absence of tool that supporting refactoring by using the KDM specification in current integrated development environments. The catalogue presented by Fowler et al.~\cite{refactImpro} provided a basis on which developers could rely to build tool support for object-oriented refactoring: similar catalogue for the KDM specification are likely to bring similar benefits to assist software engineers during the modernization process.

We found that there is a lack of utilization of ADM and it's metamodels in some software engineering methods, such as Sofware Product Line (SPL), Frameworks, Software Architecture, so on. For instance, we argue that as ADM is based on MDD - and there are research fields related to use MDD with SPL~\cite{4335247} - someone can investigate how to modernize SPL, Frameworks, and Software Architecture by means of KDM. Finally, we also noticed that all researches have been conducted in an isolated manner. For instance, there are both researches which focus on modernizing legacy systems to SaaS~\cite{5741334, SMR:SMR582} and ones that aim to modernize legacy systems Web-Services~\cite{delCastillo:2009:PRP:1529282.1529753, ICEISPerez:CastilloGCP12} but none research has been conducted in order to combine efforts and techniques.