Recent proposals in ADM have focused mainly on providing approaches to modernize legacy system to another plataform/architecture resulting in appearance of several proposals, see Section~\ref{ssub:approach}. However, if we look at overall problem of the integration of modernization into an ADM context, there is still a significant work to be done. Following the ADM approach, models are the main focus for visualizing an executable view of the system and are also used to modernize a system. Therefore in order to integrate ADM's metamodels into this larger context, the area of discovering knowledge, i.e., parsers, needs more attention along with solution to verification of models. Very few works have been reported on in the literature (e.g.,~\cite{5440163,Bruneliere:2010:MGE:1858996.1859032}) so far that have addressed parsers to represent instances of KDM, but even these parsers provide limited infrastructure to represent all KDM's layers. Thus, argue that new researches must be conducted to create a complete parser in order to represent all KDM's layer. Furthermore, the discovering of knowledge are often mostly static in a sense that they are unable to get knowledge during the executing of the target legacy system. In~\cite{5871783} the authors presented an approach that dynamically discovery knowledge of the targe system during its execution. However, this approach lacks support for complex discovery of knowledge once it is just based on the KDM's event metamodel package. Hence further research is required to enhance the discovering of knowledge, e.g., use the knowledge dynamically identified combined with other models of KDM as the code and data models in order to obtain more meaningful models, since these models make it possible to consider additional sources of embedded knowledge.
 

